%%%%%%%%%%%%%%%%%%%%%%%%%%%%%%%%%%%%%%%%%
% Friggeri Resume/CV
% XeLaTeX Template
% Version 1.2 (3/5/15)
%
% This template has been downloaded from:
% http://www.LaTeXTemplates.com
%
% Original author:
% Adrien Friggeri (adrien@friggeri.net)
% https://github.com/afriggeri/CV
%
% License:
% CC BY-NC-SA 3.0 (http://creativecommons.org/licenses/by-nc-sa/3.0/)
%
% Important notes:
% This template needs to be compiled with XeLaTeX and the bibliography, if used,
% needs to be compiled with biber rather than bibtex.
%
%%%%%%%%%%%%%%%%%%%%%%%%%%%%%%%%%%%%%%%%%

\documentclass[]{friggeri-cv} % Add 'print' as an option into the square bracket to remove colors from this template for printing

\begin{document}

\header{victor}{ sena molero}{} % Your name and current job title/field

%----------------------------------------------------------------------------------------
%	SIDEBAR SECTION
%----------------------------------------------------------------------------------------

\begin{aside} % In the aside, each new line forces a line break
\section{contact}
Rua Xavier de Almeida, 1135 apt. 192
São Paulo, São Paulo, 04211-001
Brazil
~
+055 (11) 985 511 911
+055 (11) 5062 4849
~
\href{mailto:victorsenam@gmail.com}{victorsenam@gmail.com}
\social{http://facebook.com/victoorsena}{facebook}{victoorsena}
\social{http://github.com/victorsenam}{github}{victorsenam}
\social{http://codeforces.com/profiles/victorsenam}{codeforces}{victorsenam}
\section{languages}
portuguese (native)
english (fluent)
\section{programming}
C/C++
JavaScript (and Node)
PHP
Ruby (on Rails)
CSS3 \& HTML5
Java
\end{aside}

%----------------------------------------------------------------------------------------
%	EDUCATION SECTION
%----------------------------------------------------------------------------------------

\section{education}

\begin{entrylist}

%------------------------------------------------

\entry
{2014--current}
{Universidade de São Paulo}
{Bachelor of Computer Science}

\entry
{2011--2013}
{Liceu de Artes e Ofícios de São Paulo}
{Secondary School}

\entry
{2011--2013}
{Liceu de Artes e Ofícios de São Paulo}
{Technical School of Multimedia}

%------------------------------------------------

\end{entrylist}

%----------------------------------------------------------------------------------------
%	WORK EXPERIENCE SECTION
%----------------------------------------------------------------------------------------

\section{experience}

\begin{entrylist}

%------------------------------------------------

\entry
{2013}
{TriForce Soluções Digitais}
{Software Development Internship}
{Took part in the development of web sites for companies and of a generic content managing system which is still in use by the company. Worked mainly with PHP, Javascript and HTML5/CSS3. \\
Some of the websites I've worked on:
\begin{itemize}
\item \href{http://www.gazitbrasil.com.br/}{Gazit Brasil (http://www.gazitbrasil.com.br/)}
\item \href{http://www.mmdermatologia.com.br/}{Clínica Dra. Márcia Monteiro (http://www.mmdermatologia.com.br/)}
\end{itemize}}

\entry
{2012}
{Independent Web-Development}
{Freelancer}
{I worked as a freelancer web developer for about a year with a colleague. Since it was my first professional experience with programming it was really interesting since I've learned many things about web development, dealing to clients and working in group. \\
We've developed three commercial web sites during this time, unfortunatenly none of them are still available online.}

\end{entrylist}

%----------------------------------------------------------------------------------------
%	AWARDS SECTION
%----------------------------------------------------------------------------------------

\section{extracurricular}

\begin{entrylist}

%------------------------------------------------

\entry
{2014-current}
{International Collegiate Programming Contest}
{Contestant}
{I'm currently strongly engaged on practicing for the ACM-ICPC competition and other programming contests.}

\entry
{2009}
{International Statistical Literacy Competition}
{Finalist}
{I was one of the finalists on the \href{http://iase-web.org/islp/Competitions.php?p=Competitions_2008-2009}{First International Statistical Literacy Competition} held in Durban, South Africa by the International Statistical Institute (ISI).}

%------------------------------------------------

\end{entrylist}

%----------------------------------------------------------------------------------------
%	INTERESTS SECTION
%----------------------------------------------------------------------------------------

\section{interests}

algorithms, competitive programming, web development, statistics, math, discovering interesting stuff, creating useful and cool software, music, gaming.

\end{document}
