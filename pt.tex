%%%%%%%%%%%%%%%%%%%%%%%%%%%%%%%%%%%%%%%%%
% Plasmati Graduate CV
% LaTeX Template
% Version 1.0 (24/3/13)
%
% This template has been downloaded from:
% http://www.LaTeXTemplates.com
%
% Original author:
% Alessandro Plasmati (alessandro.plasmati@gmail.com)
%
% License:
% CC BY-NC-SA 3.0 (http://creativecommons.org/licenses/by-nc-sa/3.0/)
%
% Important note:
% This template needs to be compiled with XeLaTeX.
% The main document font is called Fontin and can be downloaded for free
% from here: http://www.exljbris.com/fontin.html
%
%%%%%%%%%%%%%%%%%%%%%%%%%%%%%%%%%%%%%%%%%

%----------------------------------------------------------------------------------------
%	PACKAGES AND OTHER DOCUMENT CONFIGURATIONS
%----------------------------------------------------------------------------------------

\documentclass[a4paper,10pt]{article} % Default font size and paper size

\usepackage{fontspec} % For loading fonts
\defaultfontfeatures{Mapping=tex-text}
\setmainfont[Path = ./fonts/, 
             UprightFont = *-Regular, 
             SmallCapsFont = *-SmallCaps, 
             ItalicFont = *-Italic, 
             BoldFont = *-Bold
            ]{Fontin} % Main document font

\usepackage{xunicode,xltxtra,url,parskip} % Formatting packages

\usepackage[usenames,dvipsnames]{xcolor} % Required for specifying custom colors

\usepackage{fullpage} % Margin formatting of the A4 page, an alternative to layaureo can be \usepackage{fullpage}
% To reduce the height of the top margin uncomment: 
\addtolength{\voffset}{-1.3cm}
\addtolength{\textheight}{3cm}

\usepackage{hyperref} % Required for adding links	and customizing them
\definecolor{linkcolour}{rgb}{0,0.2,0.6} % Link color
\hypersetup{colorlinks,breaklinks,urlcolor=linkcolour,linkcolor=linkcolour} % Set link colors throughout the document

\usepackage{titlesec} % Used to customize the \section command
\titleformat{\section}{\Large\scshape\raggedright}{}{0em}{}[\titlerule] % Text formatting of sections
\titlespacing{\section}{0pt}{3pt}{3pt} % Spacing around sections

\usepackage{array}
\newcolumntype{P}[1]{>{\centering\arraybackslash}p{#1}}
\sloppy

\begin{document}

\pagestyle{empty} % Removes page numbering

\font\fb=''[cmr10]'' % Change the font of the \LaTeX command under the skills section

%----------------------------------------------------------------------------------------
%	NAME AND CONTACT INFORMATION
%----------------------------------------------------------------------------------------

\par{\centering{\Huge Victor Sena \textsc{Molero}}\bigskip\par} % Your name

%\section{Personal Data}

\begin{tabular}{rl}
\textsc{Nascimento:} & 15 de Março de 1996 \\
\textsc{Endereço:} & Rua Xavier de Almeida, São Paulo, SP, Brasil \\
\textsc{Telefone:} & +55 1 11 98551 1911\\
\textsc{email:} & \href{mailto:victorsenam@gmail.com}{victorsenam@gmail.com} \\
\textsc{github:} & \href{https://github.com/victorsenam}{victorsenam} \\
\end{tabular}

%----------------------------------------------------------------------------------------
%	EDUCATION
%----------------------------------------------------------------------------------------

\section{Educação}

\begin{tabular}{r|p{11cm}}
\textsc{Fev 2014 to Dez 2018}   & Bacharelado em Ciências da Computação na \textsc{Universidade de São Paulo}\\
\textsc{(atualmente)}              & \footnotesize{ênfase em teoria da computação} \\
\multicolumn{2}{c}{} \\
\end{tabular}


%----------------------------------------------------------------------------------------
%	WORK EXPERIENCE 
%----------------------------------------------------------------------------------------

\section{Extra Curricular}
\begin{tabular}{r|p{11cm}}
\textsc{Fev 2014}     & Competidor na \textsc{ACM-ICPC} \\
\emph{Atualmente}        & \hfill \hfill \textit{Campeão Brasileiro e Finalista Mundial} \\
    & \footnotesize{Meu time se classificou para a final mundial da ICPC de 2018 ficando em primeiro colocado na fase nacional. Atualmente, estou me preparando para esta etapa e participando de diversas competições. Utilizo principalmente C++ e estou focando meus estudos em algoritmos, estruturas de dados e geometria computacional.}\\
    & \footnotesize{Participei das escolas de verão da \textsc{ACM-ICPC} na UNICAMP em 2016, 2017 e 2018 e na USP em 2015, 2016 e 2017 assistindo a aulas em tópicos avançados ensinados por competidores bem-sucedidos de diversos países.} \\
\multicolumn{2}{c}{} \\

\textsc{Fev 2014}     & Membro do \textsc{\href{https://www.ime.usp.br/~maratona/}{MaratonIME}} \\
\emph{Atualmente}        & \hfill \hfill \textit{Grupo de estudos em programação competitiva} \\
    &\footnotesize{O grupo organiza atividades que promovem a programação competitiva e ajudam na preparação de estudantes para as competições. Eu ajudei a} \begin{itemize} \footnotesize{
        \item Desenvolver o ContestWatcherBot, um Bot de Telegram open-source feito em Node.js que avisa sobre horários de competições futuras e dá informações úteis sobre elas.
        \item Desenvolver o site do MaratonIME, um blog open-source baseado em Jekyll.
        \item Dar aulas sobre estruturas de dados, algoritmos e programação para competidores iniciantes e avançados.
    } \end{itemize} \\
\multicolumn{2}{c}{} \\

\textsc{2015 e 2016} & Hackatons \\
& \footnotesize{Participei da Facebook Hackathon em 2015 e da Facebook-Mastercard Hackathon em 2016} \\
\multicolumn{2}{c}{} \\

% \textsc{Aug 2009}     & World Finalist at \textsc{IASE ISI \href{http://iase-web.org/islp/Competitions.php?p=Competitions_2008-2009}{ISLP} 2009} \\
% & \hfill \hfill \textit{School-Level Statistics Competition} \\
% & \footnotesize{Qualified to the ISLP World Finals at Durban, South Africa in 2009. And was sponsored by my school to attend at the event, attending to lectures on descritive statistics from the World Statistics Congress.} \\
\end{tabular}

\centering{
    \begin{tabular}{P{.4\textwidth}P{.4\textwidth}}
        \href{http://codeforces.com/profile/victorsenam}{\textsc{Codeforces}}  & \href{https://www.topcoder.com/members/victorsenam}{\textsc{TopCoder}} \\
        Rating 2074, 10\textsuperscript{o} colocado no Brasil                                       & Rating 1715
        %\multicolumn{2}{c}{} \\
    \end{tabular}
}


\section{Work}

\begin{tabular}{r|p{11cm}}
\textsc{Mai-Nov 2013} & Estagiário em desenvolvimento Web na \textsc{\href{http://www.3force.com.br/}{TriForce Soluções Digitais}} \\
& \hfill \hfill \textit{Empresa de desenvolvimento Web} \\
    & \footnotesize{Desenvolvimento de uma framework de gerenciamento de conteúdo genérica, usada em todos os sites desenvolvidos pela empresa. Trabalhei principalmente com PHP, SQL, JavaScript e HTML5/CSS3 com mais dois programadores e um designer. Já que a framework seria usada em vários projetos, havia uma preocupação muito grande com a qualidade e escalabilidade do código.} \\
\multicolumn{2}{c}{} \\

\end{tabular}

%----------------------------------------------------------------------------------------
%	LANGUAGES
%----------------------------------------------------------------------------------------

\section{Programming Languages}

\begin{tabular}{rp{10cm}}
\textsc{C/C++}                 & \footnotesize{3 anos de experiência principalmente em programação competitiva e trabalhos de faculdade.} \\[0.4cm]
\textsc{JavaScript (e Node)} & \footnotesize{experiência de trabalho na TriForce. 5 anos de uso em projetos pequenos.} \\[0.4cm]
\textsc{PHP}       & \footnotesize{experiência de trabalho na TriForce.} \\
\end{tabular}

%----------------------------------------------------------------------------------------

\end{document}
